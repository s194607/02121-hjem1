\documentclass{article}
\usepackage[utf8]{inputenc}
\usepackage[T1]{fontenc}
\usepackage[danish]{babel}
\usepackage{hyperref}
\usepackage{graphicx}
\usepackage[left=4.5cm,
            top=1cm,
            right=5cm,
            bottom=5cm]{geometry}
\title{\huge Boblesortering}
\author{Patrick Clemmensen, s194706 \and Navn2, s191111 \and Navn3, s192222 \and Navn4, s194444}
\date{1. oktober 2019}

\begin{document}
\maketitle



\begin{abstract}
Dette dokument omhandler boblesortering. Der beskrives algoritmen og præsenteres en kompleksitetsanalyse
\end{abstract}

\section{Introduktion}
Boblesortering (eng. \textit{bubble sort}) er en populær \textit{sorteringsalgoritme} og er en af de simpleste algoritmer at forstå og implementere. Dog er den ikke en særlig effektiv sorteringsalgoritme\footnote[1]{Mere om dette i “Algoritmer og Datastrukturer 1”}; hverken for store eller små lister, og den anvendes sjældent i praksis. Boblesortering sorterer, som navnet antyder, elementerne i en liste ved at \textit{boble} hvert element gennem listen til sin rette plads i listen.

\subsection{Pseudokode}
Wikipedia {\cite{2}} giver følgende pseudokode for boblesortering.

\begin{verbatim}
procedure bubbleSort( A : list of sortable items ) defined as:
  do
    swapped := false
    for each i in 0 to length(A) - 2 inclusive do:
      if A[i] > A[i+1] then
        swap( A[i], A[i+1] )
        swapped := true
      end if
    end for
  while swapped
end procedure
\end{verbatim}
En illustration af en kørsel af boblesortering fra Wikipedia kan ses på figur 1.


\section{Analyse af boblesortering}
Antallet af sammenligninger, som boblesortering udfører på en tabel af længde n, er i værste fald
\begin{equation}
  \sum \limits_{i = 1}^{N-1} 1+2+3+\cdots+n-1 = \frac{n(n-1)}{2}.
\end{equation}
I bedste fald er antallet \textit{n-1}. Se tabel 1.
\newpage




\begin{thebibliography}{}
\bibitem{1}
    Donald Knuth. The Art of Computer Programming, Volume 3. Addison-Wesley.
\bibitem{2}
    \href{https://en.wikipedia.org/wiki/Bubble_sort}{https://en.wikipedia.org/wiki/Bubble\_sort}
\end{thebibliography}

\end{document}